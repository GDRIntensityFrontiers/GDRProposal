
\section{Introduction}

The remarkable success of the "standard model" (SM) of particle physics in describing the particles and their strong, electromagnetic and weak interactions has nevertheless certain limitations. For example, it does not explain dark matter, dark energy, the hierarchy of the fermion masses or the matter-antimatter asymmetry in the universe. There is a general consensus in the physics community that a theory more fundamental than the standard model should exist, which is sometimes referred to as "new physics" (NP). 

There are broadly two categories of searches for new physics: the energy frontier and the intensity frontier.  In the former, experiments are designed to try to produce (and detect) TeV-scale particles directly, i.e. via collisions at high energy. This is the main approach currently followed by the general purpose detectors, ATLAS and CMS, at the LHC. 
In order to discover new particles, one is required to run at higher energies;  such experiments are therefore said to be probing the energy frontier.
Instead, in particle physics at the intensity frontier one probes new physics not by pushing the energy scale but rather the experiment's luminosity.
This could provide signs of new physics in two ways. Either one measures SM processes for which theoretical predictions with uncertainties well under control  exist, and can be compared with  the experimental measurements:  observing a significant discrepancy between the two would be the sign of new physics. This technique is often applied to study processes which are mediated at leading order by loop diagrams. This allows yet undiscovered particles, with masses beyond the energy of the collisions, to intervene, modifying the rates and the properties of the decay respect to the SM predictions. These measurements need to be extremely precise, so they require a large quantity of data.  Apart from being a way to discover new physics, this approach  also provides constraints and highlights on the nature of the new physics and eventually on its flavour structure. Alternatively one searches for processes which don't exist in the SM, and therefore a measurement automatically signifies NP. This could either probe (effective) couplings which do not exist in the SM, or particles at scales much below the energy frontier but which have not so far been seen due to the fact that they are very weakly interacting with SM particles. For example lepton flavour violating decays, axion searches or neutrinoless double beta decay.

Experimentally, the challenge with this approach is to collect a large and pure enough data sample in order to obtain evidence of NP interactions. Theoretically, it is crucial to have the description of the processes in the SM under control. Theory and experiment then need to come together to correctly interpret the experimental results in terms of theoretical predictions, to combine all the bounds produced in the different searches which hopefully eventually will lead towards the discovery of the new physics. 

Historically the French community has been very active in this effort. From the experimental point of view, the focus today is on the  LHCb experiment, dedicated to flavor physics and currently challenging the standard model predictions with many precise measurements; it's scope extends well beyond the realm of $B$ physics. Worldwide, other experiments currently adopt a similar approach (NA62, MEG), some will start their data taking soon (Belle2) and other are in the preparatory phase (SHIP). From the theoretical side, those working in this field are very active both in the interpretation of current data in terms of new physics models and in the improving the precision on theoretical  predictions for key observables.  Given the need to compare the theoretical predictions with the experimental measurements, the interplay between theory and experiment in this field is essential. As a natural need of sharing competences and knowledge, during past years some collaborations have already risen between members of the two communities. A well known example of a fruitful exchange is the CKMfitter collaboration, which originated as a result of a French initiative. More recently, in the context of the study of rare $B$ meson decays, three  CNRS PEPS-PTI (Projet Exploratoire Premier Soutien de Physique Theorique et ses Interfaces) of one year each were proposed and accepted:  NouvPhyLHCb in 2014 and PhenoBas in 2015 and 2016. These grants permitted the organization of three fruitful workshops,  allowing to establish first connections and collaborations between LHCb experimentalists and theorists  working on $b \to sll$ transitions. 

Following discussions among people active in the field, the need for a GDR in physics at the intensity frontier has been established. In fact, we believe that the more stable framework of a GDR would be beneficial to those working on high energy physics who are focused on the intensity frontier, in order to bring this community together and reinforce the interplay between the different research lines in the field. The role of the GDR would be to 
facilitate the collaboration between different laboratories and between theorists and experimentalists, with the purpose of keeping the community in touch and informed about the latest advancements in the field, exchanging ideas and spreading knowledge. In this way the GDR will stimulate the emergence of common projects within the French community and allow it to grow. It will be a way to provide greater visibility of this large community on a national and international level. In addition, we believe that there is a real need to come together in order to discuss how research plans for the future should be shaped; including the decision of which experiments to become involved in.


We envisage the GDR to be divided into several working groups, which would both function independently and together. We have identified the following topics where there is currently activity and interest in the French community:
\begin{itemize}
\item {\bf CP violation.}  Since the $B$-factories, CP violation in the quark sector has also been proven to be a precise test of the Standard Model, through the measurement of the Unitarity Triangle. This measurement has room for improvement, and LHCb and Belle2 will provide further insight on it, as well as additional tests involving the  $B_s$ and $b$ baryons. 
\item {\bf Rare, radiative and semi-leptonic decays.} Generally mediated by loops, these decays are a powerful probe of new physics, provided that precise theoretical predictions can be made for experimentally clean observables. The large dataset collected by the LHCb experiment is currently showing the most exciting signs of slight deviations from the theoretical predictions that certainly deserves to be further analysed and deeply understood. 
\item {\bf Heavy flavour production and spectroscopy.} Not only is this field an ideal framework to test the QCD predictions, but it provides crucial input needed for other measurements and interpretations in the search for physics beyond the SM.  It further has recently revealed that quarks can form more complex structures than previously believed, i.e.~tetraquarks and pentaquarks; the existence of these bound states has now been established though they are not yet fully understood. 
\item {\bf Charm and kaon physics.} The study of kaons and charmed mesons has been at the origin of the flavor physics. Given the present experimental opportunities, a renewed interest in the analysis of their decays is emerging, as they provide complementary ways to search for new physics effects.   Although for the charm physics there is already a large production of data, for the kaons some experimental challenges need to be faced and additional theoretical observables are being proposed.  
\item {\bf Lepton flavour and the interplay with quark flavour. } 
\item {\bf Future experiments.} It will be very beneficial for our community to discuss the future of our field, at a time where future upgrades of the LHCb experiment as well as new experiments are being proposed. This GDR could play a role in identifying the priorities for French involvement in order to continue to play an active role in the future. 
\end{itemize}
In the subsequent sections of this document we will provide a brief description to highlight  the interest of the working group topics,  summarising their  current status and the proposed near-future work of the French community. 

The GDR would function through carefully planned workshops. More specifically, we plan to organize a general kick-off meeting, to bring the whole community together in order to define and consolidate collaborations and goals.  
This would be followed by a series of working group meetings, more intimate and focused on specific themes. These smaller meetings would allow detailed discussions and brainstorming within the specific topic of the working group, allowing close collaborations to emerge by really working together. Regularly, global workshops involving all the members of the GDR will be organized, where more general talks and discussions will be held and where we will ensure to share the advancements of the working groups and to address the connexions between them. This is particularly important since there is a clear interplay  between the different working groups. For example the charm and kaon physics will have to address specific experimental and theoretical issues of the field, but the results obtained will certainly have to be interpreted in the global CP violation picture and in relation with the other rare decays studies.  One of the purpose of the GDR will be also to have a wider look into what is done in the same field in other countries or in experiments where the French community is not currently directly involved, but which still represent an interest for the field. Presenting ourselves as a unified community, we will aim in establishing productive interactions inviting occasionally speakers from other experiments and other particle physics GDRs, ensuring in this way that we keep the connection with the whole field.

The format of these meetings would be free, and decided according to the specific needs and objectives, and it would be encouraged for younger members of the community to participate in the organisation and the discussion.
In fact, we further hope to use the GDR as an opportunity to put the younger members of the community in the spotlight. One way to do so will be  by giving to the postdocs the responsibility of organising and chairing the meetings. In addition, we will create an environment where PhD students would feel confident to present their work and interact with physicists from other laboratories.










