
\section{Introduction}

The need for a GDR in physics at the intensity frontier has been established, in order to bring the community together, exchange ideas and knowledge.
\!
The role of this GDR would be to stimulate the emergence of common projects within the French community, facilitating the collaboration between different laboratories and between theorists and experimentalists. 
\!
This would provide greater visibility of this large community on a national level.
\!
There is a need to come together in order to discuss how research plans for the future should be shaped; including the decision of which experiments to become involved in.
\!
The GDR would function through carefully planned workshops, both for the entire group where more general talks and discussions could be held and specialised subgroups allowing close collaborations to emerge.
\!
Topics dedicated to the interplay between different subgroups could also be beneficial on specific topics such as lepton and quark flavour or rare B decays and searches for hidden particles.
\!
We further hope to use these workshops as an opportunity to put the younger members of the community in the spotlight, by giving postdocs the responsibility of organising and chairing some meetings, and by creating an environment where PhD students feel confident to present their work and interact with physicists from other laboratories.

We envisage the GDR to be divided into several subgroups which would both function independently and together:
\begin{itemize}
\item Rare, radiative and semi-leptonic decays
\item CP violation/charmless hadronic decays
\item Heavy flavour production and spectroscopy
\item Charm and Kaon physics
\item Lepton Flavour (Neutrinos/Taonic/muonic decays/(g-2))
\item Future experiments BelleII, FCC, (SHIP)
\end{itemize}
In the subsequent sections we will provide a brief introduction and motivation to these themes as well as summarise the past, current and the proposed near-future work of the French community.


More specifically, we plan to organize a general kick-off meeting, to bring the whole community together in order to define and consolidate collaborations and goals.
\!
This would be followed by a series of smaller meetings, more intimate and focused on specific themes. The format of these meetings would be free, and decided according to the specific needs and objectives, and it would be encouraged for younger members of the community to participate in the organisation and the discussion.
\!
The emphasis on having smaller meetings would be to allow detailed discussions and brainstorming within a specialised area.
\!
We believe that the GDR structure would be beneficial to those working on high energy physics who are focussed on the intensity frontier in both consolidating the existing activities and prompting the creation of new projects and collaborations.