
\section{Introduction}

The remarkable success of the "standard model" (SM) of particle physics in describing the particles and their strong, electromagnetic and weak interactions has nevertheless certain limitations. For example, it does not explain dark matter, dark energy, the hierarchy of the fermion masses or the matter-antimatter asymmetry in the universe. There is a general consensus in the physics community that a theory more fundamental than the SM should exist, which is sometimes referred to as "new physics" (NP). The NP is expected to arise and be seen at higher energies, i.e., exploring shorter distances.

There are broadly two categories of searches for NP: the energy frontier and the intensity frontier.  In the former, experiments are designed to try to produce and consequently detect TeV-scale particles directly, i.e. via collisions at high energy.  
Such experiments are therefore said to be probing the energy frontier. This is the main approach currently followed by the general purpose detectors, ATLAS and CMS, at the LHC. 
Instead, in particle physics at the intensity frontier, which will be the focus of this proposal,  one probes NP not by pushing the energy scale but rather the experiment's luminosity.

The intensity frontier could provide signs of NP in two ways. The first one is  measuring SM processes for which theoretical predictions with uncertainties well under control  exist:  observing a significant discrepancy between the experimental measurement and the prediction  would be the sign of NP. This technique is often applied to study processes which are mediated at leading order by loop diagrams. In such diagrams, yet undiscovered particles, with masses beyond the energy of the collisions, could  intervene, modifying the rates and the properties of the decay respect to the SM predictions. These measurements need to be extremely precise, so they require a large quantity of data.  The second way is searching for processes which are hugely suppressed or forbidden  in the SM, and therefore a measurement automatically signifies NP. This could either probe (effective) couplings which do not exist in the SM, or particles at scales much below the energy frontier but which have not  been seen so far due to the fact that they are very weakly interacting with SM particles. Some examples are lepton flavour violating decays, axion searches or neutrinoless double beta decay.

Apart from being a way to discover NP, the intensity frontier approach  also provides constraints and highlights on the nature of the NP and eventually on its flavour structure. 
In fact, regardless of the experimental strategy which will eventually lead to the discovery of NP, measurements at the intensity frontier are necessary complements to the discovery itself, as they allow to  identify the theory beyond the new phenomena  through its quantum fingerprints. Additionally, it is likely that this strategy will succeed only through a diversity of measurements.

From the experimental point of view, the challenge with the intensity frontier is to collect a large and pure enough data sample in order to obtain evidence of NP interactions. A detailed understanding of the detectors features and sophisticated analysis techniques are needed to provide a large efficiency for the signal reconstruction and a powerful rejection of the backgrounds.  Historically the French community has been very active in this domain, participating to the conception and realization, as well as to the analysis of the collected data, of very successful experiments like, for example NA48 and BaBar.  The focus of the French community today is on the  LHCb experiment, dedicated to flavour physics and currently challenging the SM predictions with many precise measurements; its scope extends well beyond the realm of $B$ physics. Worldwide, several other experiments currently search for NP using high-intensity facilities (notably NA62, MEG), some will start their data taking soon (for example Belle II) and other are in the preparatory phase (for example SHIP and COMET). 

From the theory side, it is crucial to have the description of the processes in the SM under control. For example, hadronic effects need to be evaluated precisely using various advanced tools, like lattice calculations, effective field theory, sum rules.  
The French theorists working in this field are very active both in the interpretation of current data in terms of NP models and in improving the precision on theoretical  predictions for key observables. 

Given the need to compare the theoretical predictions with the experimental measurements, the interplay between theory and experiment in this field is essential. Theory and experiment need to come together to correctly interpret the experimental results in terms of theoretical predictions, and to combine all the bounds produced in the different searches, which hopefully  will lead towards the discovery of the NP. 
 As a natural need of sharing competences and knowledge, during past years some collaborations have already risen between members of the two communities. A well known example of a fruitful exchange is the CKMfitter collaboration~\cite{Charles:2004jd}, which originated as a result of a French initiative. More recently, in the context of the study of rare $B$ meson decays, four  CNRS PEPS-PTI (Projet Exploratoire Premier Soutien de Physique Theorique et ses Interfaces) of one year each were proposed and accepted: "Flagship measurements at LHCb: pursuing precision as a means to discovery" in 2012, "NouvPhyLHCb" in 2014 and "PhenoBas" in 2015 and 2016. These grants permitted the organization of  fruitful workshops,  allowing to establish first connections and collaborations between LHCb experimentalists and theorists  working on $b \to sll$ transitions. 

This interplay is fundamental for the success of the intensity frontier approach, and it needs to be further promoted.  
Following discussions among people active in the field, the need for a GDR in physics at the intensity frontier has been established. The French community proposing this GDR is made up of 61 physicists belonging to 14 different laboratories of the IN2P3 and INP institutes, and one CEA institute, with consolidated partnerships with Universities. They are listed as proponents of this document. The GDR community will be actually much larger, as it will certainly include PhD students and postdocs working in team with the proponents.  We believe that the framework of a GDR is necessary for those working on high energy physics who are focused on the intensity frontier, given the current context of particle physics. The successful program of LHCb is pointing to possible hints of NP that should be further investigated sharing the competences. In addition, these hints are mostly observed in the lepton flavour universality tests of $B$ meson decays, so that an approach mixing the quark and lepton sectors is needed.    
 The role of the GDR will be  to bring the French intensity frontier community together,  reinforcing the interplay between the different research lines in the field. It will facilitate the collaboration between different laboratories and between theorists and experimentalists, with the purpose of keeping the community in touch and informed about the latest advancements in the field, exchanging ideas and spreading knowledge. In this way the GDR will stimulate the emergence of common projects within the French community and allow it to grow. It will be a way to provide greater visibility of this large community on a national and international level. In addition,  we are in a era where new experiments are starting and other being proposed. We believe that there is a real need for experimentalists and theorists in France to come together in order to discuss how research plans for the future should be shaped, including the decision of which experiments to become involved in.

We envisage the GDR to be divided into several working groups, which would both function independently and together. We have identified the following topics where there is currently activity and interest in the French community:
\begin{itemize}
\item {\bf CP violation.}  Since the $B$-factories, CP violation in the quark sector has also been proven to be a precise test of the SM, through the measurement of the parameters of the CKM matrix. This measurement has room for improvement, and LHCb and Belle II will provide further insight on it, as well as additional tests involving the  $B_s$ meson and $b$ baryons. 
\item {\bf Rare, radiative and semi-leptonic $B$ decays.} Generally mediated by loops, these decays are a powerful probe of NP, provided that precise theoretical predictions can be made for experimentally clean observables. The large dataset collected by the LHCb experiment is currently showing the most exciting signs of slight deviations from the theoretical predictions that certainly deserves to be further analysed and deeply understood. 
\item {\bf Charm and kaon physics.} The study of kaons and charmed mesons has been at the origin of the flavour physics. Given the present experimental opportunities, a renewed interest in the analysis of their decays is emerging, as they provide complementary ways to search for NP effects.   Although for the charm physics there is already a large production of data, for the kaons some experimental challenges need to be faced and additional theoretical observables are being proposed.   
\item {\bf Heavy flavour production and spectroscopy.} Not only is this field an ideal framework to test the QCD predictions, but it provides crucial inputs needed for other measurements and interpretations in the search for physics beyond the SM.  It further has recently revealed that quarks can form more complex structures than previously believed, i.e.~tetraquarks and pentaquarks; the existence of these bound states has now been established though they are not yet fully understood.  
\item {\bf Interplay of quark and lepton  flavour. } Flavour violation in the charged lepton sector is a clear sign of NP by itself, and many experiments are directly searching for it. In addition, given the fact that at the moment some of the most interesting deviations from the SM are observed in  lepton flavour universality tests in $B$ meson decays, an approach mixing the quark and lepton sectors and combining measurements and theoretical advancements in both the field is mandatory. 
\item {\bf Future experiments.} It will be very beneficial for our community to discuss the future of our field, at a time where future upgrades of the LHCb experiment as well as new experiments are being proposed. This GDR could play a role in identifying the priorities for French involvement in order to continue to play an active role in the future. 
\end{itemize}
In the subsequent sections of this document we will provide a brief description to highlight  the interest of these  topics,  summarizing their  current status and the proposed near-future work within the GDR. 

The GDR will be leaded by two physicists, one theorist and one experimentalist, with the role of promoting a successful collaboration between the two communities. They will be helped by a "comit\'{e} scientifique"  for orienting the scientific activity of the GDR, with representatives from each working group in charge of steering and monitoring the activities,   and by a "conseil de groupement" for organization matters, with members from the different participating institutes.   
  
  The GDR would function through carefully planned workshops. The format of these meetings would be  decided according to the specific needs and objectives.  More specifically, we plan to organize a general kick-off meeting, to bring the whole community together in order to define and consolidate collaborations and goals.   This would be followed by a series of working group meetings, more intimate and focused on specific themes. These smaller meetings would allow detailed discussions and brainstorming within the specific topic of the working group, allowing close collaborations to emerge by really working together. Regularly, global workshops involving all the members of the GDR will be organized, where more general talks and discussions will be held and where we will ensure to share the advancements of the working groups and to address the connections between them. This is particularly important since there is a clear interplay  between the different working groups. For example the charm and kaon physics will have to address specific experimental and theoretical issues of the field, but the results obtained will certainly have to be interpreted in the global CP violation picture and in relation with the other rare decays studies.  There might be overlaps with other GDRs (Neutrino, Terascale) on some topics, and so we plan to organise common sessions with them to address these specific issues.  
One of the purposes of the GDR will be also to have a wider look into what is done in the same field in other countries or in experiments where the French community is not currently directly involved, but which still represent an interest for the field. Presenting ourselves as a unified community, we will aim in establishing productive interactions inviting occasionally speakers from other experiments, ensuring in this way that we keep the connection with the whole field. 

In addition, we will work to promote the emergence of a young and dynamic generation of physicist working in the field and educated in France.  In fact, we further hope to use the GDR as an opportunity to put the younger members of the community in the spotlight.  They will be encouraged to participate in the organisation and the discussion and it will be given to the postdocs the responsibility of organizing and chairing the meetings, whenever possible. In addition, we plan to  organize  a school on "Introduction and Modern developments in flavour physics" for young M2 and PhD students, and we will work to create an environment where PhD students would feel confident to present their work and interact with physicists from other laboratories. 

We believe that this proposal is in line with the history and the scientifique policy of our institutes and we are certain that  the establishment of the GDR of the intensity frontier will be a further step  to  consolidate and develop  the  productivity and the international recognition of the French community working in the field.  
