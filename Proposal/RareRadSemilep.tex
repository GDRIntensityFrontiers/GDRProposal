%\documentclass[12pt,epsf,amssymb,qsymbols]{article}
%\usepackage{tabularx}
%\usepackage{array}
%\usepackage{graphics}
%\usepackage{graphicx}
%\usepackage{psfrag}
%\usepackage{epsfig}
%\usepackage{amsmath}
%\usepackage{amssymb}
%\usepackage{ulem}
%%\usepackage{figlatex}
%\usepackage{rotating}
%\usepackage{colortbl}
%\usepackage{tabularx}
%\usepackage{longtable}
%\usepackage{multirow}
%\makeatletter
%
%
%%%%%%%%%%%%%%%%%%%%%%%%%%%%%%% Textclass specific LaTeX commands.
%\usepackage{verbatim}
%%\usepackage{citesort}
%
%\setlongtables
%
%%%%%%%%%%%%%%%%%%%%%%%%%%%%%%% User specified LaTeX commands.
%%###################################################
%%###################################################
%%######## D E F I N I T I O N S ####################
%%###################################################
%%###################################################
%\setlength{\oddsidemargin}{0pt}
%\setlength{\textwidth}{16.2cm}
%\setlength{\topmargin}{-0.35in}
%\setlength{\textheight}{22.6cm}
%\newcommand{\msbar}{{\overline{\rm MS}}}
%\newcommand{\ri}{{\rm RI-MOM}}
%\newcommand{\csw}{c_{\mbox{\scriptsize \rm SW}}}
%\newcommand{\bea}{\begin{eqnarray}}
%\newcommand{\eea}{\end{eqnarray}}
%\newcommand{\beq}{\begin{equation}}
%\newcommand{\eeq}{\end{equation}}
%\newcommand{\ec}{\end{center}}
%\newcommand{\bc}{\begin{center}}
%\newcommand{\gev}{{\rm GeV}}
%\newcommand{\mev}{{\rm MeV}}
%\newcommand{\lr}{\leftrightarrow}
%\newcommand{\pdir}{p\kern -5.2pt\raise 0.2ex\hbox {/}}
%\newcommand{\one}{1\hspace*{-1.05mm} \hbox {I}}
%\newcommand{\vdir}{v\kern -5.75pt\raise 0.15ex\hbox {/}}
%\newcommand{\kdir}{k\kern -5.75pt\raise 0.15ex\hbox {/}}
%\newcommand{\epsdir}{\epsilon\kern -5.0pt\raise 0.15ex\hbox {/}}
%\newcommand{\bvdir}{\bar{v}\kern -5.75pt\raise 0.15ex\hbox {/}}
%\newcommand{\Ddir}{D\kern -7.75pt\raise 0.20ex\hbox {/}}
%\newcommand{\Adir}{A\kern -7.75pt\raise 0.20ex\hbox {/}}
%\newcommand{\ldir}{l\kern -5.0pt\raise 0.2ex\hbox{/}}
%\newcommand{\varepsdir}{\varepsilon\kern -5.5pt\raise 0.15ex\hbox{/}}
%\newcommand{\vare}{\varepsilon}
%\newcommand{\etc}{{\it etc}}
%\newcommand{\cs}{{\cal S}}
%\newcommand{\cb}{{\cal B}}
%\newcommand{\nf}{{N_{\rm f}}}
%\newcommand{\kkbar}{K^0-\bar K^0}
%\def\bpi{$B \rar \pi \ell \nu$}
%\def\bk{B_K}
%\newcommand{\m}[0]{\phantom{$-$}}
%\newcommand{\z}[0]{\phantom{0}}
%\newcommand{\n}[0]{\cellcolor[gray]{0.85}}
%\def\ds{\displaystyle}
%\def\negcdot{\negmedspace\cdot\negmedspace}
%\newcommand{\nn}{\nonumber}
%\makeatother
%
%\begin{document}
% 
%\setcounter{footnote}{0}
%%%%%%%%%%%  Section 1
\section{Rare, radiative and semileptonic $B$ decays}

The LHCb collaboration has produced a large set of results related to the exclusive $b \to s\ell \ell $ decay modes and their results are currently dominating the field. 
In the special case of the $\cb(B_s\to \mu\mu)$, the CMS collaboration is also significantly contributing and after combining the two results, 
it turned out that the long searched $\cb(B_s\to \mu\mu)$ is only slightly lower than, but compatible with, the value
predicted in the Standard Model (SM). The $\cb(B_d\to \mu\mu)$ decay has also been seen, its branching ratio is also compatible with the value predicted by the SM. 

\par
Joint work between experimentalists ans theorists have allowed to identify an ensemble of observables which are
at the same time sensitive to the couplings to different possible sources of New Physics (NP) and as immune as possibie to non factorizable QCD effects. 
In this framework, after comparing the experimental values of those observables, as well as of $\cb(B\to K \mu\mu)$ and $\cb(B_s\to \phi \mu\mu)$, with the theoretical estimates derived in the SM, 
one finds considerable discrepancies (of the order of 3 to 3.5 standard deviations). 
It appears, however, that the most significant discrepancies occur near the charm production threshold, a region notoriously difficult for theoretical 
description of these decays because it requires an accurate estimate of the hadronic matrix element of a non-local operator corresponding to disconnected $c\bar c$-diagrams which 
cannot be computed by means of numerical simulations of QCD on the lattice. For that reason, as of now, it is not clear whether the current discrepancies are due to the lack of theoretical 
control of the $c\bar c$ contributions, or they indicate the presence of NP couplings. If the second option is adopted, the angular observables of $B\to K^\ast \mu\mu$ and $B_s\to \phi \mu\mu$ 
decay modes provide very stringent constraints on the scenarios of physics BSM. It is important to note that these results were confirmed for the $B^0\to K^\ast \mu\mu$  analysis by the BELLE collaboration. 
On top of the very rich set of results involving muons, LHCb has also performed an angular analysis  of the $B^0\to K^\ast e e $ decay mode in the low dilepton invariant mass region. The results found there are in agreement with SM but currently quite limited in statistics.  

Another experimental result which has also provoked some interest in the flavor physics community is that $R_K = \cb(B\to K\mu\mu)/\cb(B\to Kee)_{{\rm low}-q^2}$ was found to be $2.6\sigma$ smaller than  predicted in the SM, which suggests the violation of universality of the coupling to leptons (LFUV). Such a puzzling phenomenon should be scrutinized with higher statistics and tested in other similar situations, 
such as $R_K$ at high-$q^2$'s, $R_{K^\ast , \Lambda^{(\ast )}}$ at both low- and high-$q^2$'s. This observation is adding to an already noted problem of $R_{D^{(\ast)}}=  \cb(B\to D^{\ast}\tau\nu_\tau)/\cb(B\to D^{\ast}\mu\nu_\mu)$ for which the experimental result, first measured at the $B$-factories and then confirmed at LHCb, is $(2\div 4)\sigma$ larger than predicted in the SM. There are very few phenomenologically viable theoretical scenarios of NP which can simultaneously explain that $R_K^{\rm exp}<R_K^{\rm SM}$ and that $R_{D^{(\ast )}}^{\rm exp} > R_{D^{(\ast )}}^{\rm SM}$. To further understand the origin of the LFUV one can envisage doing the angular analysis of all the mentioned decay modes, and from the ratios of angular observables check whether or not a similar size of the LFUV is indeed observed. 
Furthermore, to facilitate a comparison with theory it is more sound to compare $B_s\to D_s^{(\ast)}\ell \nu_\ell$ because the theoretical uncertainty related to the chiral extrapolation in the light valence quark on the lattice is completely avoided in this way. Moreover, the emission of soft photons can differently affect $B^- \to D^{0 (\ast)}\ell^- \bar \nu_\ell$ and $B^0 \to D^{- (\ast)}\ell^+ \nu_\ell$, the modes which are usually averaged. Such a problem is much less significant of one works with $B_s\to D_s^{(\ast)}\ell \nu_\ell$ decays. 

Most of the models pretending to describe the LFUV effects allow for the lepton flavor violation (LFV) too. For that reason it is of great interest to measure the LFV modes such as $B_s\to \mu \tau$, $B\to K^{(\ast )}\mu \tau$,  $B_s\to \phi \mu \tau$, which can now be probed thanks to the large statistics achievable at the LHC. Experimental bounds on $\cb(B_s\to \mu e)$ and $\cb(B_s\to K^{(\ast )} \mu e)$ can be greatly improved thanks to the unprecedented statistics of the LHC data. These results can be very useful for phenomenology of the LFV decays and for the bigger picture that could ultimately lead to a theory of flavor of quarks and leptons. 


The work of this part of GDR will be carried out within two working groups (theory and experiment) and the outcome of their works and discussions will be presented at the annual workshops that 
will unite both the theorists and experimenters and which will be organized following the agenda described below.
\begin{enumerate}
\item Year One: Workshop on the LFUV in $B$ and $B_s$ decays\\ 
During this workshop the theorists will discuss a general scenario of NP, in an effective field theory approach, and isolate the observables which are most sensitive 
to the couplings to the vector (scalar) and/or axial (pseudoscalar) operators. Experimenters and theorists will elaborate on the feasibility of the distribution of $B_{(s)}\to D_{(s)}^\ast  \ell \nu_\ell$ 
according to the polarization of the outgoing vector meson. Furthermore, a contact with other leptonic observables should be made in order to test several plausible scenarios of NP which 
result in LFUV.

\vskip .6cm 

\item Year Two: Workshop on the angular distribution of various decay modes\\ 
With the new and more accurate experimental data it becomes mandatory to assess the hadronic uncertainties on the theory side. Lattice QCD and the QCD sum rule practitioners will try and 
evaluate the size of theoretical errors and discuss the appropriate methodology on how to account for various sources of systematic uncertainties. 
An other interesting subject could be the angular analysis of $B^0\to K^\ast \tau\tau$ and the study of new observables taking into account the direct access to the $tau$ polarization.
The study of $b \to s \ell \ell$ transition in b-baryons  is quite new and the identification of interesting observables for the $\Lambda_b \to \Lambda^\ast \mu\mu$ may be interesting.  
Possible phenomenological ideas on how to relate the hadronic quantities in several decay modes will be discussed as they might be helpful in cancelling a large part of hadronic uncertainties.
Ideas on how to treat the non-resonant $c\bar c$-contributions would be very welcome. Participants will also address the question {\it ``Which physics BSM?"}

\vskip .6cm 

\item Year Three: Workshop on the lepton flavor violation in $b$-decays\\ 
Revisiting the LFUV problem: discuss the new constraints on  $R_{K^{(\ast )}, D^{(\ast )}, \Lambda^{(\ast )}}$ obtained in Belle~II, and attempt drawing more accurate conclusions concerning the 
new physics scenarios. Focus then on the LFV modes and on interpretation of the results found by LHCb. The issues related to the identification of $\tau$ in the final state should be revisited. 
Work on the package of codes that would include all possible constraints relevant to the LFV at low and high energy and see what are the lessons one can learn about the Yukawa sector from the data. 

\vskip .6cm 

\item Year Four: Workshop on the relation to Higgs \\ 
Assess the current situation concerning the extraction of the Yukawa couplings from experimental data. In what way those data can be related to the low-energy physics observables and 
$b$-decay observables in particular. In order to address the issue of ``{\it Which theory of flavor?}" we will try and combine the searches made at Belle~II with those made at NA62 and KOTO experiments. 
Address the issue of (in)compatibility of the conclusions found in the Yukawa sector through the low-energy experiments with the LHC findings at the TeV-scale. 

\end{enumerate}
%\end{document}
