\input{shortcuts}

\section{CP Violation}
\label{sec:cpv}
Since its discovery in the kaon system, the study of CP violation has been intriguing. Great efforts have been put into understanding it, as,  apart from the interest of the phenomenon by itself and of its relations with the matter/antimatter asymmetry, it is a very powerful probe for NP.  In fact,  CP violation is particularly interesting because, in the SM, it originates from a single parameter. All the CP violating observables in the $K$, $D$ and $B$ meson sectors are thus directly related, and studying them in combination provides a highly powerful test of the SM.
Most models of New Physics are far less restrictive than the SM with respect to CP violation and allow for a
plethora of new sources. 
Most of the delicate interplays between observables
expected in the SM will therefore no longer hold in the presence of new dynamics at the TeV scale.
%None of the delicate interplays between observables expected in the SM should survive to the presence of new dynamics at the~TeV scale.
In this respect, one
activity of the French community, the CKM fitter collaboration, is precisely
to test the coherence of CP violation measurements. It is embodied in the well
known Unitary Triangle (UT), a consequence of the unitarity of the CKM matrix
in the SM. In the absence of New Physics, its sides and angles as determined
from various observables all have to agree for the triangle to close.


When experimentally testing SM predictions, it is fundamental that the
theoretical precision matches the experimental accuracy. A priori, this appears
challenging because CP violation is a purely hadronic phenomenon in the SM,
originating from the quark couplings. However, dedicated strategies have been
designed and CP-violating observables are actually among our best windows to look through when searching for  NP. For
example, in CP-violating asymmetries, most of the uncertainties cancel between
the numerator and the denominator, so that we can construct measurable
quantities with small uncertainties. Alternatively, some observables are
predicted to be so small that they can be considered forbidden in the SM, and
simply observing a non-zero value would be a clear indication of the presence of new particles.

The French community has been deeply involved in CP-violation dedicated experiments for
many years (CPLEAR, NA48, BaBar, LHCb), particularly in building and maintaining key
elements, like trigger systems, calorimeters and particle identification detectors. In addition, it has acquired expertise in several powerful techniques needed to study CP-violation: amplitude analyses, tagged-time-dependent
angular analyses, flavour tagging, neutral objects reconstruction.  Let us describe
briefly the current status and some activities planned in the near future for CP-violation measurement in the $B$ sector and in other observables. The specific case of CP-violation in the $D$ and $K$ sectors will be discussed in section~\ref{sec:CharmAndKaon}. 

\subsubsection*{CP-violation in the $B$ sector}
More specifically, there has been a great deal of involvement in France in the measurement of the unitarity angles
$\alpha$, $\gamma$ and $\phi_{s}$. Among other decay channels, the following
have been studied in detail:
$B_{(s,d)}\to D^{0} K^{*0}$, $B_{(s,d)}\to\bar{D}^{0} hh$, $B^{0}_{s} \to
D_{s} K$, $B^{0}_{s} \to J/\psi\phi$,
$B^{0}_{s} \to J/\psi\bar{K}^{*0}$. 
At the moment the focus is on the analysis of the data coming from the Run2 of LHC, particularly on the measurement of the $\phi_{s}$ angle and on the study of the charmless $b$-hadron decays. 


The measurement of $\phi_{s}$ is one of the most important goals of LHCb experiment. The value of $\phi_{s}$ is precisely predicted in the Standard Model and sets the scale for the difference between properties of matter and antimatter for Bs mesons. The predicted value is small and therefore the effects of New Physics could change its value significantly. The $\phi_{s}$ measurement with Run1 data from LHCb has been obtained analyzing $B^{0}_{s} \to J/\psi K K$ and $B^{0}_{s} \to J/\psi \pi \pi$ decays. It  is the most precise to date, and is shown on Figure~\ref{figphis} combined with the measurements of other experiments. With the increasing precision that will be obtainined using the Run2 data of LHCb, and even more the data coming from the upgrade that will lead to an error of the order of 0.01 rad, it will become fundamental to control the sub-leading contributions coming from penguin diagrams. These diagrams are doubly Cabibbo suppressed respect to the tree diagrams, nevertheless their contribution has not yet been precisely estimated in QCD.  The effort to control this contribution will require a strong collaboration with the theorists and eventually exploring new approaches. 

% $B^{0}\to\rho\rho$, $B_{(s,d)}\to K^{0}_{\mathrm{S}} hh$, $B^{0} \to K^{0}_{\mathrm{S}} K \pi$,

Another field in which the experimental French groups have been  very active in the B-factories era, and still are nowadays in LHCb, is the CP-violation related studies of charmless $b$-hadron decays.  These decays have a number of theoretical applications and provide a probe to new physics. For instance, the decays \BdtoKsPiPi and \BdtoKsKK are dominated by \btoqqbars ($q = u,d,s$) loop transitions.  Mixing-induced \CP asymmetries in such decays are predicted to be approximately equal to those in \btoccbars transitions, \eg $\Bd\to\jpsi\KS$, by the
Cabibbo-Kobayashi-Maskawa mechanism~\cite{Cabibbo:1963yz,Kobayashi:1973fv}.
However, the loop diagrams that dominate the charmless decays can have
contributions from new particles in several extensions of the Standard Model,
which could introduce additional weak phases~\cite{Buchalla:2005us,Grossman:1996ke,London:1997zk,Ciuchini:1997zp}.
A time-dependent analysis of the three-body Dalitz plot allows measurements of
the mixing-induced \CP-violating
phase~\cite{Dalseno:2008wwa,Aubert:2009me,Nakahama:2010nj,Lees:2012kxa}. 
 The current experimental measurements of \btoqqbars decays~\cite{HFAG} show
fair agreement with the results from \btoccbars decays (measuring the weak
phase \Pbeta) for each of the scrutinised \CP eigenstates.
There is, however, a global trend towards lower values than the weak phase
measured from \btoccbars decays.
The interpretation of this deviation is complicated by QCD
corrections, which depend on the final state~\cite{Silvestrini:2007yf} and
are difficult to handle.
An analogous extraction of the mixing-induced \CP-violating phase in the
\Bs system ($\phi_s$) will, with a sufficiently large dataset, also be possible with
the \BstoKsKPi decay, which can be compared with that from, \eg
$\Bs\to\jpsi\phi$.

The charmless three-body analyses provide a long-term physics program that can profit from the LHCb upgrade. In fact, these analyses proceed in increasingly complex steps, which become more and more sensitive to new physics observables with the growing dataset, and with more observed decay modes. One of the long-term goals is to perform full flavor- and time-dependent Dalitz-plot analyses of the \BstoKshhp modes to measure the weak phases $\beta$ and $\phi_s$. Moreover, with the upgrade of LHCb, more modes, eventually with more neutral hadrons, are being considered. Recent theoretical and experimental activities have focused on the determination of the CKM angle \Pgamma from charmless $B$ meson decays using and refining the methods proposed in Refs.~\cite{Ciuchini:2006kv,Gronau:2006qn,Bhattacharya:2013cla}.  




%+ PCI40, SciFi
%CPLEAR, NA48, NA62, BaBar, LEP(?) LHCb

%In the coming years, most of the \textcolor{red}{French} experimental effort will go into the analysis of data collected by  LHCb during run2 and in its upgrade plans. 

%One of the biggest challenge will be to store and analyse the enormous quantity of data from the LHC.

%CHRISTOPHER>
%From the theory side ... (CKMfitter, QCD challenge, precision challenge,
%sub-leading diagram estimation, penguin pollution) {\bf This paragraph should be developed} ...






\subsubsection*{CP-violation in the $D$ and $K$ sectors}

For a complete picture of CP violation, and to test the CKM paradigm, CP
violation in $K$ and $D$ physics should be studied in parallel to that in $B$
physics. Details will be discussed in section~\ref{sec:CharmAndKaon}. 


\subsubsection*{CP-violation in other observables}

In the SM, CP violation is a purely flavoured phenomenon, arising from the
presence of three families of matter particles. This partly explains the
strong suppression in physical observables, and thereby their high sensitivity
to non-standard sources of CP violation. At the same time, this feature is not
fully understood and raises several questions:

\begin{itemize}
\item CP violation via the strong interaction is mysteriously absent from the
SM. If present, it would deeply alter the picture, in particular for electric
dipole moments. This is the so-called strong CP problem. In this context, the
French community is actively involved in the next generation of neutron EDM experiments.

\item The study of CP violation could have deep cosmological consequences. For
example, one solution to the strong CP problem involves a new particle, the
axion, the relic density of which could play a role in the context of dark matter.
Another puzzle is the origin of the baryon asymmetry of the Univerve, which
seems to require some new sources of CP violation.
\end{itemize}

Thus, exploring CP violation in light mesons has implications well beyond the
strict context of flavor physics, and may shed new lights on some of the most
fundamental puzzles.





\subsubsection*{Plans for the GDR}
In the next five years, the GDR will provide an opportunity to continue the measurements started many years ago, using the full run2 dataset of the LHC, and to explore new routes.  In summary:
\begin{itemize}
\item the effort towards the measurement of $\phi_{s}$ will continue, with a focus on controlling the sub-leading penguins contributions;
\item several additional ways to measure $\gamma$ and $\alpha$ will be explored, using for example charmless $B$ decays, in order  to overconstrain the UT;
\item we will have the opportunity to explore CP violation in baryons, copiously produced at the LHC, and this is currently a mostly unexplored field, complementary to the meson field;
\item  the first results  of the Belle II experiment will be discussed, and  their complementarity with the LHCb results will be assessed;
\item CP violation in charm,  still to be discovered, will be searched for;
\item the theoretical advancement in CP violation in kaon, like  lattice studies of the $\varepsilon^{\prime}$ matrix elements, will be followed, together with the achievements of the  NA62 experiment.
%, currently in data taking and aiming to measure the  $K^{+}%
%\rightarrow\pi^{+}\nu\nu$ branching fraction.
\item the CP violation results in B physics will be put in relation with the CP violation results in the charm and kaon sector; 
\item the GDR will be the forum for  brainstorming on future upgrade of the LHCb experiment (2025-2035), as well as on future CP violation experiments in general.
%\item the theoretical advancement in CP violation in kaon, like  lattice studies of the $\varepsilon^{\prime}$ matrix elements, will be pursued, together with the achievements of the  NA62 experiment, currently in data taking.
%and aiming to mesure the  $K^{+}\rightarrow\pi^{+}\nu\nu$ branching fraction.
%\item EDM: should we include the nEDM experiment? It is among the IN2P3 projects {\bf Yes, this sentence should be developed}.
\end{itemize}

For all the above points, the synergy between experimental and theoretical communities is essential, because no major discovery can be made if theoretical uncertainties and experimental errors are not under control. Advances in controlling
hadronic effects, for example using lattice QCD simulations or analytic tools like sum rules, will help to improve the situation.
 The new data coming from the LHC and soon from Belle2, as well as from NA62 will certainly result in important advancements in the CP violation field, and we need to ensure that we contribute to this effort and work towards the correct interpretation of these measurements in the global CP violation picture. 

%It should be noted that there is a clear interplay of this working group with the We will also actively participate to the brainstorming on future upgrade of the LHCb experiment, i.e. plans for the 2025-2035 period.

\begin{figure}[!htb]
\begin{center}
\includegraphics[width=7.5cm]{hfag_Spring2016_DGsphis_zoom.pdf}
\includegraphics[width=7.5cm]{rhoeta_small_global.pdf}
\end{center}
\caption{Left plot: 68\% CL regions in $B^{0}_{s}$ width difference $\Delta\Gamma_{s}$
and weak phase $\phi_{s}$ obtained from individual and combined CDF, D0,
ATLAS, CMS and LHCb likelihoods of $B^{0}_{s}\to J/\psi\phi$, $B^{0}_{s}\to
J/\psi KK$, $B^{0}_{s}\to J\psi\pi\pi$ and $B^{0}_{s}\to D_{s}^{+}D_{s}^{-}%
$. The expectation within the Standard Model is shown as the black rectangle. Right plot: the current status of the CKM unitarity triangle test from the CKMFitter collaboration. }%
\label{figphis}%
\end{figure}



%
%\section{Charmless $b$ Hadron Decays}
%\label{sec:charmless}
%
%The study of of $b$-hadron decays to hadronic final states with no charmed particles allow for a rich array of studies. A few examples are the measurements of branching fractions, CP asymmetries, weak and strong phases and the CKM angles; they probe the dynamics of weak and strong interactions. The typical branching fractions of these modes are below $10^{-5}$ and thus their analyses are feasible only with large data samples and the use of powerful tools to reject background. The LHCb experiment is an adequate experimental environment for these analyses, offering the possibility to study decays of light $B$ mesons, $B_s$ mesons and $b$ baryons.
%
%In particular, CP-violation related studies of charmless $b$-hadron decays have a number of theoretical applications and that provide a probe to new physics, along the lines described in Sec.~\ref{sec:cpv}.
%For instance, the decays \BdtoKsPiPi and \BdtoKsKK are dominated by \btoqqbars ($q = u,d,s$)
%loop transitions.
%Mixing-induced \CP asymmetries in such decays are predicted to be approximately
%equal to those in \btoccbars transitions, \eg $\Bd\to\jpsi\KS$, by the
%Cabibbo-Kobayashi-Maskawa mechanism~\cite{Cabibbo:1963yz,Kobayashi:1973fv}.
%However, the loop diagrams that dominate the charmless decays can have
%contributions from new particles in several extensions of the Standard Model,
%which could introduce additional weak phases~\cite{Buchalla:2005us,Grossman:1996ke,London:1997zk,Ciuchini:1997zp}.
%A time-dependent analysis of the three-body Dalitz plot allows measurements of
%the mixing-induced \CP-violating
%phase~\cite{Dalseno:2008wwa,Aubert:2009me,Nakahama:2010nj,Lees:2012kxa}. 
% The current experimental measurements of \btoqqbars decays~\cite{HFAG} show
%fair agreement with the results from \btoccbars decays (measuring the weak
%phase \Pbeta) for each of the scrutinised \CP eigenstates.
%There is, however, a global trend towards lower values than the weak phase
%measured from \btoccbars decays.
%The interpretation of this deviation is made complicated by QCD
%corrections, which depend on the final state~\cite{Silvestrini:2007yf} and
%are difficult to handle.
%An analogous extraction of the mixing-induced \CP-violating phase in the
%\Bs system ($\beta_s$) will, with a sufficiently large dataset, also be possible with
%the \BstoKsKPi decay, which can be compared with that from, \eg
%$\Bs\to\jpsi\phi$.
%
%An impressive harvest of results from charmless hadronic $B$ mesons decays was obtained by the $B$ factories. Several french groups participated in these studies within the BaBar experiment, and in particular, a part of the present LPNHE-LHCb group. The LHCb experiment is already playing an important role in this area of physics, with the participation of the LPC and LPNHE groups. Both have contributed to the LHCb analysis of the decay modes \BstoKshhp ($\had^{(\prime)} = \pi, K$) with $1$~fb$^{-1}$ of data, which are being pursued with $3$~fb$^{-1}$.  In particular they are performing amplitude analyses (aka Dalitz-plot analyses) of \BdtoKsPiPi and \BdtoKsKK decays. At LHCb, the first step of the charmless $b$ hadron decays physics programme is to establish the signals of yet unobserved rare modes. The only yet-unobserved \BstoKshhp mode is \BstoKsKK. The LPC group also performs analyses of $B_s \to \rho^0 \rho^0$, and $\Lambda_b^0 (\Xi_b^0) \to p \had \had^{\prime}\had^{\prime\prime}$ decays.
%
%All the analyses mentioned above provide a long-term physics programs that can profit from the LHCb upgrade. These analyses proceed in increasingly complex steps, which become more and more sensitive to new physics observables with the growing dataset, and with more observed decay modes. One of the long-term goals is to perform full flavor- and time-dependent Dalitz-plot analyses of the \BstoKshhp modes to the measure the weak phases $\beta$ and $\beta_s$. Recent theoretical and experimental activity has focused on the determination of the CKM angle \Pgamma from charmless $B$ meson decays using and refining the methods proposed in Refs.~\cite{Ciuchini:2006kv,Gronau:2006qn,Bhattacharya:2013cla}. The LPNHE group is checking the applicability of the method described in the last reference to the LHCb physics analysis. Moreover, with the upgrade of LHCb, more modes, eventually with more neutral hadrons, are being considered.
