\section{Conclusion}

The intensity frontier is a strategic approach to search for new physics: historically, many of the discoveries in high-energy physics came first as indirect evidence in high-intensity experiments, and only afterwards were confirmed by direct, targeted searches. The intensity frontier is, furthermore, a domain in which the French particle physics community has been traditionally very competitive.

In addition, and interestingly enough, tantalizing hints of beyond-SM effects exist in data from recent and present experiments at the intensity frontier, among the others LHCb, the $B$-factories, and experiments having measured the anomalous magnetic moment of the muon. More data on all of these discrepancies are expected to come from the new generation of experiments that are starting or being planned.

Theoretical progress will ensure the theory predictions to error-match the experimental accuracy of the planned experiments. Furthermore, additional clean observables have been proposed, and more are under investigation for the LHCb upgrade, for Belle II, and also for other flavour experiments outside $B$-physics, for example NA62, as well as for experiments aimed at new light-particle searches.

In short, we are in the favorable circumstance of interesting data flowing from experiments, more data expected to come, and of a French community with more than the critical size and the international reputation to be competitive in these searches and their interpretation. We therefore consider timely and strategic to form the "GDR of the Intensity Frontier". This  will allow for a financially well-defined structure to pursue collaborations within the community, beneficial among the other things to strengthen the interaction between the experimental and theoretical parties involved. Furthermore, the "GDR of the Intensity Frontier" will be the place to share our experience and our knowledge, reinforce existing bounds and inspire new collaborations, thereby ensuring that the French community stays competitive, and continues to focus on the most promising topics of the field. Finally, it will provide a forum to discuss the future of the domain, and naturally promote the emergence of a young and dynamic generation of physicists active in the field, and educated in France. This latter point will be crucial to transmit the heritage of our community and to consolidate its international competitiveness over time.
