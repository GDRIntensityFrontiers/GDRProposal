\section{Conclusion}
%The intensity frontier is a strategic approach to search for new physics.  Its validity is recognized at an international level and  it is a domain in which the French particle physics community is traditionally very active. Interesting and puzzling results are being produced by the experiments currently taking data, and further more are expected to come from the new generation of experiments that are starting or being planned. Theoretical progress are ensuring the precision of the currently performed tests, and additional clean observables are under investigation. 

%The French community working in this field recognizes the need of coming together to pursue these searches together with a renewed enthusiasm  and with a stronger collaboration between the experimental and the theoretical laboratories in France.   
%The GDR intensity frontier will be the place where we will be able to put together our experience, share our knowledge, renforce bounds and inspire new collaborations, ensuring that the French community continues to be competitive and focused on the most appealing topics of the field. It will additionally provide a forum to discuss the future of the field, and naturally promote the emergence of a young and dynamic generation of physicists. All this will allow to keep the current involvement and acquire an even higher visibility at a national and international level.

The intensity frontier is a strategic approach to the search for new physics: historically, many of new particles discovered in high-energy physics were first observed via indirect evidence at high-intensity experiments; only later were these confirmed by direct, targeted searches. The intensity frontier is, furthermore, a domain in which the French particle physics community has traditionally been internationally reknowned.

In addition, and interestingly enough, tantalizing hints of beyond-SM effects exist in data from recent and present experiments at the intensity frontier, among others LHCb, the B-factories, and experiments having measured the anomalous magnetic moment of the muon. More data on all of these discrepancies are expected to come from the new generation of experiments that are underway or being planned.

Theoretical progress is needed to ensure that the uncertainty on predictions to matches the experimental accuracy of the planned experiments. Furthermore, additional clean observables have been proposed and more are under investigation for the LHCb upgrade, the Belle upgrade, other flavor experiments outside B-physics e.g.~NA62, and experiments aimed at new light-particle searches.}

In short, the circumstances are currently favourable in that both is there a large influx of data from experiment with more expected to come, and a French community with a more than critical size and an international reputation in performing and interpreting these searches. We therefore consider it to be timely and strategic to form a GDR entitled "the Intensity Frontier" (GDRInF). This  will provide the well-defined (financial) structure necessary to pursue collaborations within the community, which amongst other things would be beneficial via strengthening the interaction between the experimental and theoretical parties involved. Furthermore, GDRInF will provide a platform to share our experience and knowledge, reinforce existing bounds and inspire new collaborations, thereby ensuring that the French community continues to produce competitive research, and to focus on answering the most relevant questions in the field. Finally, it will provide a forum to discuss the future of the field, and naturally promote the emergence of a young and dynamic generation of physicists active in the field, and educated in France. This latter point will be crucial to transmit the heritage of our community and to consolidate its international competitiveness over time.
