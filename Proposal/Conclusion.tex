\section{Conclusion}
%The intensity frontier is a strategic approach to search for new physics.  Its validity is recognized at an international level and  it is a domain in which the French particle physics community is traditionally very active. Interesting and puzzling results are being produced by the experiments currently taking data, and further more are expected to come from the new generation of experiments that are starting or being planned. Theoretical progress are ensuring the precision of the currently performed tests, and additional clean observables are under investigation. 

%The French community working in this field recognizes the need of coming together to pursue these searches together with a renewed enthusiasm  and with a stronger collaboration between the experimental and the theoretical laboratories in France.   
%The GDR intensity frontier will be the place where we will be able to put together our experience, share our knowledge, renforce bounds and inspire new collaborations, ensuring that the French community continues to be competitive and focused on the most appealing topics of the field. It will additionally provide a forum to discuss the future of the field, and naturally promote the emergence of a young and dynamic generation of physicists. All this will allow to keep the current involvement and acquire an even higher visibility at a national and international level.

\textcolor{red}{The intensity frontier is a strategic approach to search for new physics: historically, many of the discoveries in high-energy physics came first as indirect evidence in high-intensity experiments, and only afterwards were confirmed by direct, targeted searches. The intensity frontier is, furthermore, a domain in which the French particle physics community has been traditionally very competitive.}

\textcolor{red}{In addition, and interestingly enough, tantalizing hints of beyond-SM effects exist in data from recent and present experiments at the intensity frontier, among the others LHCb, the B-factories, and experiments having measured the anomalous magnetic moment of the muon. More data on all of these discrepancies are expected to come from the new generation of experiments that are starting or being planned.}

\textcolor{red}{Theoretical progress will ensure the theory predictions to error-match the experimental accuracy of the planned experiments. Furthermore, additional clean observables have been proposed, and more are under investigation for the LHCb upgrade, for the Belle upgrade, for other flavor experiments outside B-physics, for example NA62, and for experiments aimed at new light-particle searches.}

\textcolor{red}{In short, we are in the favorable circumstance of interesting data flowing from experiments, more data expected to come, and of a French community with more than the critical size and the international reputation to be competitive in these searches and their interpretation. We therefore consider timely and strategic to form a "GDR Intensity Frontier". This  will allow for a financially well-defined structure to pursue collaborations within the community, beneficial among the other things to strengthen the interaction between the experimental and theoretical parties involved. Furthermore, the "GDR Intensity Frontier" will be the place to share our experience and our knowledge, reinforce existing bounds and inspire new collaborations, thereby ensuring that the French community stays competitive, and continues to focus on the most promising topics of the field. Finally, it will provide a forum to discuss the future of the field, and naturally promote the emergence of a young and dynamic generation of physicists active in the field, and educated in France. This latter point will be crucial to transmit the heritage of our community and to consolidate its international competitiveness over time.}
